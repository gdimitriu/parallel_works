\section{Crearea retelei si transmiterea matricii}

\hspace{5mm}Deoarece am ales pentru impartirea matricii in linii ciclice, 
versiunea prezentata in sectiunea \ref{partitionare-matrice}, trebuie
 sa realizam o retea in inel intre sistemele care conlucreaza.

Dupa ce se citeste, cu ajutorul functiei {\it int readconfig()} fisierul de
configurare, avem structurile de date {\it adr} care contin adresa sistemului 
respectiv si portul pe care se va face comunicarea cu clientul. Aceste adrese 
sunt organizate intr-un vector de structuri in ordinea citirii, primul sistem 
din acest vector va fi sistemul care va initializa conexiunea in inel.

Functia {\it int configurare\_retea()} care este implementata in fisierul
{\bf comunicatii\_client.h} implementeaza configurarea retelei, adica
realizarea unei retea cu conexiuni in inel intre servere si fiecare server
este conectat pe un port separat cu procesul master. De remarcat ca
clientul este un client concurent pe cind serverele in acesta etapa nu sunt
concurente.

Realizarea retelei implica fazele:
\begin{enumerate}
\item Stabilirea legaturii intre procesul master si toate serverele.
\item Transmiterea adreselor IP (anterioara si urmatoare) si porturilor.
\item Realizarea conectarilor propriu-zise.
\end{enumerate}

\subsection{Protocolul de transmitere a adreselor}

\hspace{5mm}Deoarece este o retea in inel fiecare server nu trebuie sa se
conecteze decit cu serverul urmator si cel precedent. Pentru serverul
urmator el este un client pe cind pentru serverul anterior este un server
(primeste conectare). Conectarea cu cele doua servere se face pe porturi
care sunt trimise de catre procesul master.
De accea procesul master trebuie in prima etapa sa transmita porturile si
numai adresa serverului urmator.

Deoarece daca avem un sistem multiprocesor putem folosi acesta facilitate
pornind mai multe servere pe acest sistem cu diferite porturi pentru
conectarea cu procesul master, nu putem avea porturi fixe pentru conectarea
cu serverul anterior si cel urmator. Ca atare porturile vor fi transmise de
catre procesul master in etapa a doua a comunicatiei.

Dupa ce a fost stabilita legatura cu toate serverele, toate socketurile pe
care se face comunicarea se introduc intr-o lista de selectie de citire 
(selectia se face cu ajutorul instructiunii {\it select}).

Serverele dupa ce realizarea conectarea si pornesc un nou proces (conform
tipologiei UNIX) transmit catre procesul master un mesaj "OK".

Intr-o bucla conditionata de transmiterea tururor datelor catre servere se
executa :

Pentru toate servele se verifica daca a venit vreun mesaj de la server pe
coada de read si daca da se executa, instructiunea {\it select} va astepta 
numai 2 secunde :

\begin{enumerate}
\item Se citeste acest mesaj (pentru a nu ramine in buffer), el este folosit
doar la sincronizare nu conteaza ce este in acel mesaj deoarece este
ignorat.
\item Se transmite zero daca este primul server si unu pentru celelalte
servere.
\item Formez un vector cu doua pozitii in care memorez cele doua porturi.
Pozitia zero este pentru serverul urmator iar pozitia unu este pentru
serverul anterior. De remarcat ca acestea pornesc de la portul 55002, de
fapt 55001 este adresa primului server si 55002 este adresa celui de-al
doilea server, adresele sunt ale serverelor din interiorul serverelor
(adresele de conectare in jos).
\item Se transmite acest vector catre server.
\item Se transmite adresa IP a serverului urmator.
\item Se receptioneaza mesajul primit de la server.Se ignora (este un mesaj
de sinconizare.
\item Se seteaza vectorul {\it ready} la pozitia serverului.
\item Scot din acest server din coada de read.
\end{enumerate}

Daca vectorului {\it ready} este numarul de servere se seteaza un flag {\it
flag\_iresire} care realizeaza conditia de iesire din bucla.

\subsection{Protoculul de realizare a inelului}

\hspace{5mm} Deoarece conexiunile prin socketuri sunt blocante si in 
implementarea de fata nu se poate face si transmie si receptie in acelasi 
timp trebuie o disciplina in realizarea conexiunilor. Acesta se poate
observa din \ref{realizare-inel} unde sunt prezentate functiile pentru
relizarea inelului in servere.
Acesta s-ar putea face daca avem doua threaduri concurente: unul pentru 
emisie si altul pentru receptie.

Disciplina in realizarea conexiunii inseamna ca serverul intii va satisface
legatura cu serverul anterior si peurma va initiliza conexiunea cu serverul
urmator. Exceptie la regula prezentata face primul server care va realiza
intii conexiunea cu serverul urmator si apoi va satisface conexiunea cu
serverul anterior. Acesta deoarece el ca in cazul unei explozii in lant este
detonatorul urmind ca prin simpatie sa se realizeze si celelalte conexiuni.
De asemenea pentru a ne lua o masura de preverede, deoarece nu stim cit de
rapide sunt conexiunile si masinile, primul server va incepere realizarea
conexiunilor numai dupa ce toate celalte servere au confirmat ca sunt in
pozitia de asteptare ({\it accept} din etapa de initializare a conexiunii prin
socketuri.

Etapa de conectare a inelului se realizeaza cu ajutorul a doua functii {\it
conect\_sv\_up()} pentru conectare catre serverul urmator care se executa
prima pentru sprimul server si a doua pentru al doilea server si {\it
conect\_sv\_dw()} pentru conectare catre serverul anterior care se executa a
doua pentru primul server si prima pentru celelalte servere.

Pentru serverele diferite de zero:
\begin{enumerate}
\item Astept mesajul "START" daca acest mesaj nu este start mai astept un
mesaj si apoi inchid socketul deoarece serverele dinaintea lui nu au pornit.
\item Pornesc functia {\it conect\_sv\_dw} care realizeaza conectarea in jos
(Adica porneste un server). 
\item Dupa ce fac {\it listen} trimit catre clientul\_master un mesaj "ACK"
care spune ca in pina aici totul a mers bine. Daca nu a mers bine se
transmite catre procesul master codul de eroare si se asteapta mesajul de
terminare a conexiunii.
\item Procesul master asculta pentru toate serverele si daca unul sau mai
multe nu trimit mesajul "ACK" ci trimit codurile de eroare trimite catre
toate serverele un mesaj "EXIT" care va inchide inelul.
\item Daca totul este bine se transmite un mesaj de pornire "START" catre
toate serverele care receptionind acest mesaj vor realiza functia accept
(blocanta).
\item In final se transmite catre primul server mesajul de pornire "START"
care incepe realizarea conexiunilor in inel.
\end{enumerate}

\subsection{Trimiterea matricii}

\hspace{5mm}Dupa ce se realizeaza conectarea in inel se citeste matricea din
fisier cu ajutorul functiei {\it read\_data} se realizeaza citirea matricei si 
impartirea ei catre servere. In final avem pentru fiecare sistem, aceste
fiind organizate intr-un vector de structuri {\it sdata} care contine: o 
matrice de marimea (nr\_variabile+1 x nr\_variabile/nr\_procesoare), un
vector de marimea (nr\_variabile/nr\_procesoare), un contor care va indica
cite linii sunt alocate de fapt sistemului deoarece matricea si vectorul sunt
supradimensionate la marimea maxima prin rotunjire superioara si o variabila
ultim care este un flag care ne spune daca acest sistem are ultima ecuatie
din sistemul de ecuatii (este folosit de catre {\bf back-substitution} si
eliminarea Gaussiana).

Acesta parte de transmitere este de asemenea concurenta si este realizata
asemanator cu transmiterea adreselor catre servere, este implementata in
functia {\it send\_data()} din fisierul {\bf comunicatii\_client.h} fiind
prezentata in sectiunea \ref{send-linii}.

Trimiterea matricii se face in doua etape:
\begin{enumerate}
\item Se transmite numarul de variabile si numarul de linii care ii revim
masinii.
\item Se transmite matricea pe linii.
\end{enumerate}

Catre server trebuie trimise urmatoarele date in prima etapa: numarul de variabile
(necunoscute) si numarul de linii pentru care se face prelucrarea (acest
numar poate diferi de la o masina la alta in functie de cum se poate imparti
numarul de variabile la numarul de servere pe care le avem la dispozitie) si
bineinteles matricea care ii revine.

Intr-o bucla conditionata de transmiterea tuturor datelor primei faze se
executa:

Pentru toate serverele se verifica daca a venit vreun mesaj pe coada de read
si daca da se executa pentru serverul respectiv:
\begin{enumerate}
\item Citesc acest mesaj pentru a nu ramine in coada, el este ignorat.
\item Transmit numarul de necunoscute (varibiabile) care este egal cu
numarul de linii si coloane ale matricii.
\item Transmit numarul de linii de prelucrat care ii revin serverului
respectiv. Acest numar de obicei este diferit de la un server la altul
deoarece de obicei nu se poate imparti corect matricea la numarul de servere
pe care le avem.
\item Transmit care masina este ultima ca ecuatii (care masina are ultima
ecuatie din sistem, masina de la care se va porni back-substitution).
\item Scot din coada de read serverul respectiv si setez variabila din {\it
ready}.
\end{enumerate}

Din bucla while se iese in momentul in care toate serverele au primit datele
ce ii revin testind ca suma valorilor din vectorul {\it ready} sa fie egala
cu numarul de servere.

In acest moment serverele au date despre lungimea matricii si pot face
alocare dinamica pentru matrice (aici se intelege coloanele si termenul
liber).

Dupa ce se face alocarea matricii serverele initiaza faza a doua a
conexiunii in care se cer liniile matricii.

Intr-o bucla conditionata de transmiterea tuturor liniile matricii se
executa:

Pentru toate serverele se verifica daca a venit vreun mesaj pe coada de read
si daca da se executa pentru serverul respectiv:
\begin{enumerate}
\item Citesc acest mesaj pentru a nu ramine in coada, el este ignorat.
\item Trimit liniile alocate serverului, una cite una cu ajutorul functiei
{\it write} care nu este o functie blocanta si are avantajul ca poate
trimite blocuri binare de date.
\item Trimit vectorul de linii (un vector in care se spune fiecare linie
care este in cadrul  matricii sistemului de ecuatii), se trimite tot cu
functia {\it write} deoarece stie sa transmita un vector intreg de date
float.
\item Scot din coada de read serverul respectiv si setez variabila din {\it
ready}.
\end{enumerate}

Din bucla while se iese in momentul in care toate serverele au primit liniile 
ce ii revin testind ca suma valorilor din vectorul {\it ready} sa fie egala
cu numarul de servere.

Din acest moment toate serverele au matricea de date care ii revine si poate
incepe prelucrearea liniilor prin algoritmul Gauss-Jordan.

Structurile de date pe care le avem in procesul\_master pot fi acum
dealocate deoarece nu mai avem nevoie de ele.
\section{Trimiterea datelor}

\hspace{5mm}Dupa ce sistemul a fost rezolvat rezultatele sistemului sunt in
prima masina in variabila rezultate. Aceasta functie este in fisierul
{\it comunicatii\_client} si are numele {\bf recv\_data()} pentru client si
este implementata in cadrul fisierului {\it gaussd.c}.

Cind datele sunt gata serverul 0 transmite un mesaj catre client un "0"
catre trigereaza functia {\bf recv\_data()} dupa care se transmite
rezultate.

Acum este intrebat utilizatorul daca doreste o scoatere a datelor intr-un
fisier, in special daca sitemul este cu foarte multe necunoscute. Daca nu se
 doreste se afiseaza pe ecran. Daca se doreste este intrebat de numele
fisierului pina cind aceste poate fi scris, este intrebat din nou daca apare
o eroare.

Dupa afisarea sau scoaterea intr-un fisier a datelor. Utilizatorul este
intrebat daca doreste o reintroducere a unui alt sistem de ecuatii. Nu se
recomanda pentru ca inca acest lucru nu functioneaza foarte bine.

Daca se doreste o reintroducere se trimite catre servere mesajul de "READY"
care trimit serverele in faza a doua a protocolului de comunicatie iar
clientul in faza de introducere a sistemului. Bineinteles ca se eliberaza
memoria si se va trece din nou prin faza de alocare a memoriei.

Daca se doreste iesire se trece la protocolul de inchidere a conexiune si
distrugere a inelului, protocol care este prezentat in sectiunea urmatoare.

\section{Distrugerea inelului}

\hspace{5mm}Protocolul de distrugere a inelului este implementat in
fisierul {\it comunicatii\_client.c} in functie {\it inchid\_sistem()} iar
in server in fuctiile {\bf deconectez\_up} si {\bf deconectez\_dw()}
prezentate in \ref{deconectare-inel}.

Prima etapa consta in distrugerea legaturilor dintre servere si client.
Acesta se realizeaza trimitind de la client un mesaj, "EXIT", si
retransmiterea de catre server a acestui mesaj, dupa care se inchide
conexiunea cu serverul din partea clientului. Acesta fiind un lient
concurent se realizeaza bineinteles cu ajutorul instructiunii de selectie
{\it select} la cererea serverului. Dupa acesta se dealoca partile de
memorie alocate dinamic, in client si in server, dupa care se trece la etapa
a doua a deconectarii inelului. Deci clientul a iesit iar acesta se executa
numai intre servere.

A doua etapa se executa diferentiat in functie de numarul de ordine al
serverului. Exista doua functii {\bf deconectez\_up()} si {\bf
deconectez\_dw()} care deconecteaza clientul respectiv serverul.

In toate sistemele cu exceptia primului sistem se deconecteaza intii
serverul si apoi clientul, pentru primul sistem se deconecteaza intii
clientul si apoi serverul deoarece ca la creearea inelului trebuie sa existe
un sistem care declanseaza dezactivarea in lant a inelului.

Functia de deconectare a clientului are pasii:

\begin{enumerate}
\item Trimite catre server un mesaj, "EXIT".
\item Receptioneaza de la server un mesaj, "EXIT", care este oglindirea
mesajului trimis de catre client.
\item Inchide socketul de comunicatie, clientul.
\end{enumerate}

Pentru server este aceeasi rutina ca pentru client dar in oglinda deoarece
clientul trebuie sa aiba un partener de discutie, care este serverul.

Dupa ce se deconecteaza serverul se deconecteaza si clientul, pentru primul
sistem intii se deconecteaza clientul si apoi serverul.

Cind se iese din procesul fiu serverul este gata oricind sa preia o noua
conexiune de la client.