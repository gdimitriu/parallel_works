\chapter{Introducere}

\hspace{5mm}Matricele se clasifica in doua categorii in functie de 
numarul de zerouri pe care le contin. Prima catogie sunt matricele dense sau
matrice pline, cu putine sau fara elemente zero. A doua categorie sunt
matricele  sparse sau matrice rarefiate in care majoritatea elementelor sunt
zero. Mai precis o matrice este considerata rarefiata daca in operatiile
efectuate asupra ei se poate utiliza numarul si locatia elementelor zero
pentru a reduce timpul de calcul a unei matrice dense de acceasi dimensiune.
\section{Maparea maticelor pe procesoare}

\hspace{5mm}Pentru a prelucra o matrice paralel sau distribuit trebuie sa
partitionam matricea astfel incit partitiile respective sa fie asignate
diferitelor procesoare sau masini. Partitionarea matricei are o foarte mare
importanta in algoritmii paraleli deoarece o anumita partitionare poate
influenta foarte mult timpul de executie a unui algoritm. De accea este bine
sa determinam  care schema de mapare este mai apropiata de fiecare algoritm.
Schemele complete de mapare sunt prezentate in \cite[151-168]{kumar}.

\subsection{Partitionarea in benzi orientate (Striped partitioning)}

In partitionarea in benzi orientate, matricea este divizata in grupuri de
rinduri sau coloane complete, si fiecarui procesor (sistem) ii este asignat
un asemenea grup. Partitionarea se numeste {\bf block-striped} daca fiecarui
procesor ii sunt asignate rindurile sau coloanele continuu asa cum este
prezentat in figura de mai jos a). Astfel in cazul unei matrice n x n pe p
sisteme (numerotate $P_0,P_1,...,P_{p-1}$) pe coloane, sistemul $P_i$ contine
coloanele cu indicele (n/p)i,(n/p)i+1,(n/p)i+2,...,(n/p)(i+1)-1.

\label{partitionare-matrice}
 In partitionarea {\bf cyclic-striped} se imparte matricea pe
rinduri sau coloane care sunt distribuite ciclic in sisteme cum este prezentat
 in figura de mai jos b). Astfel in cazul unei matrice n x n pe p sisteme
(numerotate $P_0,P_1,...,P_{p-1}$) pe rinduri , sistemul $P_i$ contine
rindurile cu indicii i,i+p,i+2p,...,i+n-p.

\vspace{5mm}
\includegraphics{mapare.ps}
\vspace{5mm}

Se poate partitiona si hibrid astfel se partitioneaza in blocuri continue
care se distribuie ciclic (se numeste {\bf block-cyclic-striped}. In care
matricea este impartita in blocuri de q rinduri ($q<n/p$) si aceste blocuri
sunt distribuite pe  sisteme  intr-o maniera ciclica.
\subsection{Partitionare pe blocuri ({\bf Checkeroard partitioning})}

\hspace{5mm}In partitionarea pe blocuri matricea est divizata in blocuri
mici rectangulare sau patrate sau submatrici care sunt distribuite pe
sisteme. Intr-o partitionare pe blocuri uniforma aceste blocuri au aceeasi
marime. O partitionare pe blocuri imparte si rindurile si coloanele deci
nici unui sistem nu ii revine o line sau coloana intreaga. Ca si in
partitionarea in benzi blocurile pot fi impartite in blocuri sau ciclic.
Poate fi ciclica daca mapam rindurile pe procesoare intr-o maniera ciclica
urmate de coloane. Similar putem crea un hibrid  block-cycling-checkerboard
impartind matricea in $q^2p$ blocuri de marime q x q si alocindu-le intr-o
maniera ciclica. Partitionare pe blocuri este o maniera naturala de a mapa o
matrice pe o retea bidimensionala de sisteme sau procesoare.