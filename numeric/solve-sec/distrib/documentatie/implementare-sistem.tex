\chapter{Implementare rezolvarii sistemului}
\section{Implementarea eliminarii Gauss}

\hspace{5mm}Este realizata in fisierul {\it calcule\_server.h} in cadrul
functiei {\bf gaussian\_elimination()} si este implementarea algoritmului de
eliminare Gauss pipline cu maparea matricii in rinduri cu $p<n$ procesoare
prezentat in sectiunea \ref{algoritm} dar care defapt este algoritmul
cu implementare pe n procesoare prezentat in sectiunea \ref{implementare}.

In versiunea in carea fost implementata am transmis toata linia in loc de o
parte din linie deaorece mi s-a parut mai fezabil sa folosesc o singura
instuctiune {\it write} care se stie ca este atomica decit mai multe
instructiuni {\it write} astfel se evita efecte nedorite cind se lucreaza pe
o singura masina (uniprocesor sau multiprocesor).

Protocol de transmitere:
\begin{enumerate}
\item Se receptioneaza numarul liniei.
\item Se receptioneaza numarul hostului care transmite linia.
\item Se receptioneaza linia intreaga.
\item Daca hostul urmator nu este hostul care a trimis linia sau daca
sistemul curent nu este cel care are ultima ecuatia si este la ultima
ecuatie (pentru calcul), transmite mai departe linia.
\item Se fac eliminarile pentru toate liniile ramase, nu numai pentru linia
curenta.
\item Daca mai sunt linii de receptionat pentru linia curenta (numarul linie
pentru care se face calculul -1) se reia de la pasul 1.
\item Se face calculul pentru linia curenta.
\item Cu rezultatul obtinut de la pasul alterior se fac eliminarile si
pentru liniile ramase deoarece nu se va mai primi acesta linie.
\item Se trimite numarul liniei.
\item Se trimite numarul hostului care transmite linia.
\item Se trimite linia intreaga.
\item Se reia de la pasul 1.
\end{enumerate}

Implementarea acestui protocol este prezentat in sectiunea
\ref{eliminare-imp}.

\section{Implementarea substitutiei {\bf back-substitution}}

\hspace{5mm}Dupa eliminarea gaussiana presentata in sectiunea anterioara
matricea plina a sistemului devine o matrice triunghiular superioara cu
elementele de pe diagonala principala unu, care se rezolva cu algoritmul
{\bf back-substitution} prezentat in sectiunea \ref{substitutie}.

Dintre toate versiunile de back-substitution in acest proiect a fost
implementata versiunea secventiala deoarece versiunea de implementare
pipeline cu mapare pe rinduri ciclice nu a functionat la data prezentarii
lucrarii.

Implementarea protocolului este prezentat in sectiunea \ref{substitutie-imp}
