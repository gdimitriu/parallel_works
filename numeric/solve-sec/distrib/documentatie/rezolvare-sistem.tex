\chapter{Metoda Gauss-Jordan}

\hspace{5mm}In acest sectiune discutam problema rezolvari unui sistem de
ecuatii liniar de forma:
$$\begin{array}{ccccc}
a_{0,0}x_0&+a_{0,1}x_1&+...\ldots&+a_{0,n-1}x_{n-1}&=b_0\\
a_{1,0}x_0&+a_{1,1}x_1&+...\ldots&+a_{1,n-1}x_{n-1}&=b_1\\
\vdots&\vdots&&\vdots&\vdots\\
a_{1,0}x_0&+a_{1,1}x_1&+...\ldots&+a_{1,n-1}x_{n-1}&=b_1\\
\end{array}$$

In notatia matriceala sistemul se poate scrie $Ax=b$. Unde A este o matrice
densa n x n cu coeficientii $A[i,j]=a_{i,j}$ iar b este un vector n x 1
$[b_0,b_1,\ldots,b_{n-1}]^T$ si x este vectorul solutiei dorite
$[x_0,x_1,\ldots,x_{n-1}]^T$.

Sistemul de ecuatii Ax=b este uzual rezolvat in doua etape. In prima dupa o
serie de manipulari algebrice, sistemul orginal este redus la un sistem
superior-tringular de forma:
$$\begin{array}{cccccc}
x_0&+u_{0,1}x_1&+u_{0,2}x_2&+\ldots&+u_{0,n-1}x_{n-1}&=y_0\\
&x_1&+u_{1,2}x_2&+\ldots&+u_{1,n-1}x_{n-1}&=y_1\\
&&&&\vdots&\vdots\\
&&&&x_{n-1}&=y_{n-1}
\end{array}$$

Acest sistem se poate scrie ca Ux=y, unde U este o matrice superior
triunghiulara unitara in care toate intrarile de sub diagonala principala
sunt zero iar diagonala principala este unu. Formal $U[i,j]=0\; daca\; i>j$,
altfel $U[i,j]=u_{i,j}$, iar $U[i,i]=0\; daca\; 0\le i<n$. In a doua etapa a
rezolvari sistemului de ecuatii liniare, matricea superior-triunghiulara
este rezolvata in ordine inversa pentru $x[n-1]$ la $x[0]$ cu o procedura
cunoscuta ca {\bf back-substitution}.
\pagebreak
\section{Algoritmul Gauss-Jordan}

\hspace{5mm}Acest algoritm este descris pe larg in \cite[178-196]{kumar}
\subsection{Implementarea secventiala}

\hspace{5mm}Eliminarea Gaussiana seriala are trei bucle cu teste. Exista
citeva variatii ale acestui algoritm, depinzind de ordinea in care sunt
aranjate buclele. Programul urmator este cel pe care il vom adopta pentru
implementarea paralela.
$$Programul\; 1$$
\label{program54}
\begin{enumerate}
\item {\bf procedure} GAUSSIAN\_ELIMINATION(A,b,y)
\item {\bf begin}
\item \hspace{5mm}{\bf for} k:=0 {\bf to} n {\bf do} //bucla externa
\item \hspace{5mm}{\bf begin}
\item \hspace{10mm} {\bf for} j:=k+1 {\bf to} n-1 {\bf do}
\item \hspace{15mm} $A[k,j]:=A[k,j]/A[k,k]$; // pasul diviziunii
\item \hspace{10mm} $y[k]:=b[k]/A[k,k]$;
\item \hspace{10mm} $A[k,k]:=1$;
\item \hspace{10mm} {\bf for} i:=k+1 {\bf to} n-1 {\bf do}
\item \hspace{10mm} {\bf begin}
\item \hspace{15mm} {\bf for} j:=k+1 {\bf to} n-1 {\bf do}
\item \hspace{20mm} $A[i,j]:=A[i,j]-A[i,k]\cdot A[k,j]$; //pasul eliminare
\item \hspace{15mm} $B[i]:=b[i]-A[i,k]\cdot y[k]$;
\item \hspace{15mm} $A[i,k]:=0$;
\item \hspace{10mm} {\bf endfor;}\hspace{10mm}// linia 9
\item \hspace{5mm} {\bf endfor;}\hspace{15mm}//linia 3
\item {\bf end} GAUSSIAN\_ELIMINATION;
\end{enumerate}

Acest program converteste un sistem de ecuatii liniar Ax=b catre un sistem
superior triungiular Ux=y. Se presupune ca matricea U ocupa aceeasi portiune
de memorie ca si A si este rescrisa pe portiunea superioara a lui A.
Elementul $A[k,j]$ calculat la linia 6 a programului 1 este
de fapt $U[k,j]$. Similar elementul $A[k,k]$ atribuit cu 1 la linia 8 este
$U[k,k]$. Programul 1 presupune ca $A[k,k]\ne0$ cind il utilizeaza ca
divizor in liniile 6 si 7.

In acesta sectiune ne vom concentra numai pe operatiile asupra matricii A in
programul 1. Operatiile asupra vectorului b pe liniile 7 si 13 ale
programului se implementeaza la fel. De altfel in restul sectiunii vom
ignora acesti pasi. Daca pasii din liniile 7,8, 13 si 14 nu se efectueaza
atunci programul 1 da ca factorizarea LU a matricii A ca un produs L x U.
Dupa ce s-a terminat procedura, L este stocat in partea inferioara
triunghiulara a matricii A, si U ocupa locatiile sub diagonala principala.

Pentru k variind intre 0 si n-1, procedura eliminarii Gaussiane elimina
sistematic variabilele $x[k]$ de la ecuatiile k+1 la n-1 deci de aceea
matricea de coeficienti devine superior-tringhiulara. Cum se vede din
programul 1,  in iteratia k a buclei exterioare (pornita la linia 3), un
apropiat multiplu a ecuatiei k este scazuta din fiecare ecuatie de la k+1 la
n-1 (bucla pornita de la linia 9). Multiplele ale ecuatie k (sau linii
matricii A) sunt alese astfel incit coeficientii k devin zero in ecuatiile
de la k+1 la n-1 eliminind $x[k]$ din aceste ecuatii. O implementare tipica
a procedurii eliminarii Gaussiene in iteratia k a buclei exterioare este
aratata in figura de mai jos. Iteratia k a bulei exterioare nu impune nici
un calcul  pe rindurile de la 1 la k-1  sau coloanele de la 1 la k-1. De
altfel in acest stadiu numai partea de jos-dreapta k x k a submatricii A
(portiunea comuna in figura de mai jos) este activa din punct de vedere al
calcului.

\vspace*{5mm}
\includegraphics{gauss.ps}
\vspace*{5mm}


Eliminarea Gaussiana implica $n^2/2$ diviziuni (linia 6) si aproximativ
$(n^3/3)-(n^2/2)$ scaderi sau inmultiri (linia 12). In acesta sectiune, vom
presupune ca fiecare operatiune scalara tine o unitate de timp. Cu aceast
presupunere, timpul de lucru secvential este aproximativ $2n^3/3$ pentru n
mare:
$$W-\frac{2}{3}n^3$$

\subsection{Implementare paralela cu partitionare pe benzi orientate}

\hspace{5mm}In acesta sectiune vom prezenta implementarea programului 1
paralel cu partitionare in benzi orientate pe linii. Foarte similara este si
implementarea cu partionare in benzi orientate pe coloane.

Consideram ca o linie este asignata unui singur procesor  deci matrice A n x
n este impartita la n procesoare notate $P_0,P_1,\ldots ,P_{n-1}$. In acesta
mapare procesorul $P_i$ initial stocheaza elementele $A[i,j]$ pentru $0\le
j<n$.

Programul 1 ne arata ca $A[k,k+1],A[k,k+2],\ldots, A[k,n-1]$ sunt
impartite la $A[k,k]$ (linia 6) la inceputul iteratiei k. Toate matricile
implicate in acest proces apartin unui singur procesor. Deci acest pas nu
necesita o comunicatie, in partea a doua a calcului (pasul de eliminare de
la linia 12), elementele (dupa impartire) liniei k sunt utilizate de
celelalte linii ale matricii active. Deci avem nevoie de o distribuire unul la
 toti  pentru linia k catre procesoarele care stocheaza liniile de la k-1 la
n-1. In final se face calculul $A[j,j]:=A[i,j]-A[i,k]\dot A[k,j]$ care se
suprascrie peste matricea ramasa.

Numarul de calcule in iteratia k sunt n-k-1 diviziuni pentru  procesorul
$P_k$. Similar calculele implica n-k-1 inmultiri si scaderi in iteratia k
pentru toate $P_i$ cu $k<i<n$. Presupunind ca o singura operatie aritmetica
ia un timp unic, timpul total petrecut in calcul in iteratia k este
$3(n-k-1)$. Atentie cind se realizeaza diviziunea, procesoarele ramase p-1
sunt idle, iar cind procesoarele $P_{k+1},\ldots,P_{n-1}$ fac eliminarea ,
procesoarele $P_0,\ldots,P_k$ sunt idle. Complexitatea de calcul a acestei
implementari este $3\sum_{k=0}^{n-1}(n-k-1)$ care face $\frac{3\dot
n(n-1)}{2}$.

Pasii de comunicare iau $(u_s+t_w(n-k-1))$ pe un hipercub. Deci in total
timpul de comunicare este $\sum_{k=0}^{n-1}(t_s+t_w(n-k-1))\log n$ care
este egal cu $t_sn\log n+t_w(n(n-1)/2)$. Timpul total paralel de lucru al
acestui algoritm pe hipercub este:
$$ T_p=\frac{3}{2}n(n-1)+t_sn\log n+\frac{1}{2}t_wn)n-1)\log n$$

Deoarece numarul de procesoare este n, costul, sau produsul procesor-timp,
este $(O(n^3\log n)$. Acest cost este mai mare asimptotic fata de timpul de
lucru al algoritmului secvential deci acest algoritm paralel nu este cost
optimal.

\subsection{Implementarea cu comunicatie pipeline}

\label{implementare}
\hspace{5mm}Acum vom prezenta impelementarea eliminareii Gaussiene care este
cost-optimal pe un inel de n procesoare, retea sau hipercub pentru o matrice
de coeficienti de n x n.

In algoritmul paralel prezentat in sectiunea precedenta cele n iteratii ale
buclei externe ale programului 1 sunt executate secvential. La un anumit
timp, toate procesoarele lucreaza la aceeasi iteratie. Iteratia k+1 porneste
numai dupa ce toate calculele si comunicatiile iteratiei k s-au terminat.
Performanta algoritmului poate fi imbunatatita substantial daca procesoarele
lucreaza asincron; astfel nici un procesor asteapta alt procesor pentru a
termina iteratia pentru a porni cealalta iteratie. Acesta se numeste
versiunea {\bf asicrona sau pipeline} a elimianrii Gaussiene. Figura de mai
jos ilustreaza programul 1 executat pipeline pentru o matrice 5 x 5
impartita logic pe un vector liniar de 5 procesoare.

In timpul iteratiei k din programul 4.5, procesorul $P_k$ transmite o parte
a rindului k a matricii procesoarelor $P_{k+1},\ldots,P_{n-1}$. Presupunem
ca procesoarele sunt conectate intr-un vector liniar, iar $P_{k+1}$ este
primul procesor care receptioneaza rindul k de la procesorul $P_k$. Apoi
procesorul $P_{k+1}$ trebuie sa trimita mai departe acesta data catre
$P_{k+2}$. De altfel dupa trimitrea rindului k catre $P_{k+2}$, procesorul
$P_{k+1}$ nu trebuie sa astepte pentru a incepe etapa eliminarii (linia 12)
pina cind procesoarele pina la $P_{n-1}$ au receptionat rindul k. Similar,
$P_{k+2}$ poate incepe prelucrarea imediat dupa ce a transmis rindul k catre
$P_{k+3}$ si asa mai departe. Intre timp dupa terminarea calcului de la
iteratia k, $P_{k+1}$ poate realiza etapa diviziunii (linia 6), si sa
porneasca sa trimita rindul k+1 catre $P_{k+2}$.

In eliminarea Gaussiana pipeline fiecare procesor independent realizeaza
urmatoare secventa de actiuni repetitiv pina cind cele n iteratii se
termina. Din considerente de simplicitate, presupunem ca pasii 1 si 2 iau
acelasi timp de procesare (acesta presupunere nu afecteaza analiza):

\begin{enumerate}
\item Daca un procesor are data destinata pentru alte procesoare, trimite
acesta data catre cel mai apropiat procesor.
\item Daca procesorul poate realiza prelucrari utilizind datele pe care le
are le va face.
\item Altfel, procesorul va astepta receptionarea date pentru una din
actiunile de mai sus.
\end{enumerate}

Figura \ref{5x5} prezinta cei 16 pasi in executia paralela pipeline a
eliminarii Gaussiene pentru o matrice 5 x 5 impartita pe 5 procesoare. In
figura \ref{5x5}(a) , primul pas este sa realizam diviziunea pe rindul
0 in procesorul $P_0$. Rindul 0 modificat este trimis la $P_1$ (figura 
\ref{5x5}(b)), care il retransmite la $P_2$ (figura \ref{5x5}(c). Acum $P_1$
este liber sa realizeze etapa eliminarii utilizind rindul 0 (figura
\ref{5x5}(d)). In etapa urmatoare (figura \ref{5x5}(e)),$P_2$ realizeaza
etapa eliminarii utilizind rindul 0. In acelasi timp, $P_1$, terminind
calculul pentru iteratia 0, porneste etapa de diviziune a iteratiei 1. La un
timp dat, diferite stadii ale aceleasi iteratii pot fi active pe diferite
procesoare. In figura \ref{5x5}(h) procesorul $P_2$ realizeaza eliminarea la
iteratia 1 in timp ce procesarele $P_3$ si $P_4$ sunt angajate in
comunicarea pentru aceeasi iteratie. De altfel mai mult de o iteratie poate
fi activa simultan pe diferite procesoare. In figura \ref{5x5}(i)
procesorul $P_2$ executa etapa diviziunii pentru iteratia 2 in timp ce
procesorul $P_3$ realizeaza eliminarea de la pasul 1.

\vspace*{5mm}
\label{5x5}
\hspace{20mm}Figura \ref{5x5}
\includegraphics{matrice.ps}
\vspace*{5mm}

Acum vom arata ca, spre deosebire de algoritmul sincron in care toate
procesorele sunt in acceasi iteratie intr-un timp dat, versiunea pipeline
sau asincrona a eliminarii Gaussiene este cost optimala. In figura \ref{5x5}
ne arata initializarea unor iteratii consecutive ale buclei externe a
programului 1. Un total de n iteratii sunt initiate. 
Ultima iteratie modifica numai coltul de dreapta-jos
a matricii, de altfel el este terminat intr-un timp constant dupa
initilizare. Numarul total de pasi in intrega procedura a est O(n). In
fiecare pas, impreuna O(n) elemente sunt comunicate intre procesoarele direct
conectate, sau pasul de diviziune este realizat pe O(n) elemente ale
rindului, sau etapa de eliminare este realizata pe O(n) elemente ale
rindului. Fiecare din aceste operatii iau O(n) timp. De altfel intrega
procedura consta in O(n) pasi fiecare de compexitate O(n), si rularea sa
paralela este $O(n^2)$. Deoarece sunt utilizate n procesoare costul este
$O(n^3)$, care este de acelasi ordin ca eliminarea Gaussiana secventiala.
Deci  versiunea pipeline a eliminaraii Gaussiene este cost opetimala pe un
vector liniar de procesoare. Deoarece un vector liniar de procesore poate fi
implementat pe o retea sau hipercub acest algoritm este cost optimal pentru
aceste arhitecturi.

\subsection{Impartirea pe mai putin de n procesoare}

\label{algoritm}
\hspace{5mm}Implementarea precedenta pipeline paralela a eliminarii Gaussiene 
este usor de adaptat pentru cazul in care avem $n>p$. Adica a impartirii
unei matrice n x n pe p procesoare.

Dupa cum am prezentat in sectiunea \ref{partitionare-matrice} sunt doua
tipuri de partitionare pentru impartirea pe benzi orientate: partitionare pe 
blocuri si partitionare ciclica.

In cazul partitionarii pe blocuri fiecarui procesor ii revin un numar de
rinduri continue n/p. La iteratia k algoritmul trebuie ca partea activa a
rindului k sa fie trimisa catre procesoarele care stocheaza rindurile
$K+1,k+2+\ldots,n-1$. Fiecare procesor va executa (n-k-1)n/p multiplicari si
eliminari in timpul pasului de eliminare la iteratia k. In ultimile n/p-1
iteratii nici un procesor nu are rinduri active. In implementare pipeline
numarul de operatii aritmetice la iteratia k este (2(n-k-1)n/p) este mult
mai mare decit numarul de cuvinte comunicate (n-k-1) de catre un procesor in
acceasi iteratie. De altfel pentru un n foarte mare si un p rezonabil
timpul accesat operatiilor aritmetice este cu mult mai mare decit timpul
necesar comunicatiei. Presupunind ca o inmultire scalara si o scadere ia
aceeasi unitate de timp, timpul total de rulare al acestui algoritm
(ignorind depasirile in comunicatii) este $2(n/p)\sum_{k=0}^{n-1}(n-k-1)$
care este aproximativ egal cu $n^3/p$. Deci se poate observa ca desi au fost
ignorate timpul de comunicatie timpul este cu 3/2 mai mare decit timpul de
lucru al unui algoritm secvential. Ineficenta se datareaza ca datorita
impartirii unele procesoare sunt idle, altele partial incarcate si altele
complet active.

Acesta se poate rezolva impartind matricea ciclic in care toate procesoarele
sunt incarcate la fel exceptind o linie. Deorece sunt n iteratii depasirea
in cazul unui procesor idle este numai de $O(n^2)$ data de $O(n^3)$ pentru
impartirea pe bocuri.
\subsection{Rezolvarea sistemului tringular {\bf Back Substitution}}

\label{substitutie}
\hspace{5mm}Acum vom prezenta a doua etapa din rezolvarea unui sistem de
ecuatii liniare. Dupa ce matricea plina A a fost redusa la o matrice
superior triunghiulara U cu unu pe diagonala principala, vom realiza
back-substitution pentru a determina vectorul x. Un algoritm secvential
pentru rezolvare sistemului de ecuatii Ux=y este prezentat mai jos in
programul 2.

$$Programul\; 2$$
\begin{enumerate}
\item {\bf procedure} BACK\_SUBSTITUTION(U,x,y)
\item {\bf begin}
\item \hspace{5mm}{\bf for} k:=n-1 {\bf downto} 0 {\bf do} //bucla
principala
\item \hspace{5mm}{\bf begin}
\item \hspace{10mm} $x[k]:=y[k]$;
\item \hspace{10mm}{\bf for} i:=k-1 {\bf downto} 0 {\bf do}
\item \hspace{15mm} $y[i]:=y[i]-x[k]\cdot U[i,k]$;
\item \hspace{5mm} {\bf endfor};
\item {\bf end} BACK\_SUBSTITUTION
\end{enumerate}

Pornind de la ultima ecuatie, la fiecare iteratie a buclei principale
(liniile 3-8) a programului 2 se realizeaza calcularea variabilei si
substituitia ei inapoi in ecuatiile ramase. Programul realizeaza aproximativ
$n^2/2$ inmultiri si scaderi. A se nota ca numarul de operatii aritmetice
din back-substituition este mai mic decit in eliminarea gausiana cu un
factor de O(n).

Daca consideram impartirea matricii U (n x n) pe rinduri in blocuri pe p
procesoare, iar vectorul y distribuit uniform pe procesoare. Valoarea
variabilei rezulvata intr-o iteratie tipica a buclei principale trebuie sa
fie trimisa catre celelalte procesoare. Comunicatia poate fi realizata pipeline.
Dar se  poate observa ca timpul predominant in acesta etapa este comunicatia
dintre procesoare fata de realizarea operatiilor. In fiecare iteratie a
implementarii pipeline un procesor receptioneaza (sau genereaza) valoarea
unei variabile si o trimite acesta valoare la procesorul urmator. Utilizind
valoarea variabilei primite procesorul realizeaza de asemena cele n/p
inmultiri si scaderi (liniile 6 si 7). Deci  fiecare pas al implementarii
pipeline cere un timp constant de comunicatie si O(n/p) timp de calcul.
Algoritmul se termina in O(n) pasi iar timpul total de lucru paralel este
$O(n^2/p)$.

Daca matricea este impartita in blocuri pe o retea logica de procesoare 
$\sqrt{p}\; x\; \sqrt{p}$  si elementele vectorului sunt distribuite de-a
lungul unei coloane a retelei de procesoare, numai cele $\sqrt{p}$ procesore
care contin vectorul realizeaza calcule. Utilizind comunicatie pipeline
pentru a comunica elementele apropiate ale matricii U catre procesorul care
contine elementul y corespunzator (linia 7), algoritmul poate fi executat in
timpul $O(n^2/\sqrt{p})$. De altfel costul implemnterarii paralele a
algoritmului back-substitution cu mapare pe blocuri este $O(n^2\sqrt{p})$.
Algoritmul nu est cost optimal deoarece costul secvential este $O(n^2)$.