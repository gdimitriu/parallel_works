\documentclass [12pt]{report}
\usepackage [dvips]{graphics}
\usepackage{rotating}
\usepackage [romanian]{babel}
\usepackage {pictex}
\usepackage{a4}
\title{Implementare distribuita a rezolvarii sistemelor de ecuatii}
\date{}
\setlength{\textheight}{9in}
\setlength{\textwidth}{6in}
\setlength{\oddsidemargin}{0mm}
\setlength{\evensidemargin}{0mm}
\setlength{\marginparwidth}{0mm}
\setlength{\marginparsep}{0mm}
\begin{document}

Realizarea retelei implica fazele:
\begin{enumerate}
\item Stabilirea legaturii intre procesul master si toate serverele.
\item Transmiterea adreselor IP (anterioara si urmatoare) si porturilor.
\item Realizarea conectarilor propriu-zise.
\end{enumerate}

Trimiterea adreselor:
\begin{enumerate}
\item Se citeste mesajul de la server (pentru a nu ramine in buffer), el 
este folosit doar la sincronizare nu conteaza ce este in acel mesaj 
deoarece este ignorat.
\item Se transmite zero daca este primul server si unu pentru celelalte
servere.
\item Formez un vector cu doua pozitii in care memorez cele doua porturi.
Pozitia zero este pentru serverul urmator iar pozitia unu este pentru
serverul anterior. De remarcat ca acestea pornesc de la portul 55002, de
fapt 55001 este adresa primului server si 55002 este adresa celui de-al
doilea server, adresele sunt ale serverelor din interiorul serverelor
(adresele de conectare in jos).
\item Se transmite acest vector catre server.
\item Se transmite adresa IP a serverului urmator.
\item Se receptioneaza mesajul primit de la server.Se ignora (este un mesaj
de sinconizare.
\item Se seteaza vectorul {\it ready} la pozitia serverului.
\item Scot din acest server din coada de read.
\end{enumerate}

Conectarea inelului:
Pentru serverele diferite de zero:
\begin{enumerate}
\item Astept mesajul "START" daca acest mesaj nu este start mai astept un
mesaj si apoi inchid socketul deoarece serverele dinaintea lui nu au pornit.
\item Pornesc functia {\it conect\_sv\_dw} care realizeaza conectarea in jos
(Adica porneste un server). 
\item Dupa ce fac {\it listen} trimit catre clientul\_master un mesaj "ACK"
care spune ca in pina aici totul a mers bine. Daca nu a mers bine se
transmite catre procesul master codul de eroare si se asteapta mesajul de
terminare a conexiunii.
\item Procesul master asculta pentru toate serverele si daca unul sau mai
multe nu trimit mesajul "ACK" ci trimit codurile de eroare trimite catre
toate serverele un mesaj "EXIT" care va inchide inelul.
\item Daca totul este bine se transmite un mesaj de pornire "START" catre
toate serverele care receptionind acest mesaj vor realiza functia accept
(blocanta).
\item In final se transmite catre primul server mesajul de pornire "START"
care incepe realizarea conexiunilor in inel.
\end{enumerate}

\pagebreak
Trimiterea matricelor:

\begin{enumerate}
\item Se transmite numarul de variabile si numarul de linii care ii revim
masinii.
\item Se transmite matricea pe linii.
\end{enumerate}

Catre server trebuie trimise urmatoarele date in prima etapa: numarul de variabile
(necunoscute) si numarul de linii pentru care se face prelucrarea (acest
numar poate diferi de la o masina la alta in functie de cum se poate imparti
numarul de variabile la numarul de servere pe care le avem la dispozitie) si
bineinteles matricea care ii revine.

Intr-o bucla conditionata de transmiterea tuturor datelor primei faze se
executa:

Pentru toate serverele se verifica daca a venit vreun mesaj pe coada de read
si daca da se executa pentru serverul respectiv:
\begin{enumerate}
\item Citesc acest mesaj pentru a nu ramine in coada, el este ignorat.
\item Transmit numarul de necunoscute (varibiabile) care este egal cu
numarul de linii si coloane ale matricii.
\item Transmit numarul de linii de prelucrat care ii revin serverului
respectiv. Acest numar de obicei este diferit de la un server la altul
deoarece de obicei nu se poate imparti corect matricea la numarul de servere
pe care le avem.
\item Transmit care masina este ultima ca ecuatii (care masina are ultima
ecuatie din sistem, masina de la care se va porni back-substitution).
\item Scot din coada de read serverul respectiv si setez variabila din {\it
ready}.
\end{enumerate}

Din bucla while se iese in momentul in care toate serverele au primit datele
ce ii revin testind ca suma valorilor din vectorul {\it ready} sa fie egala
cu numarul de servere.

Intr-o bucla conditionata de transmiterea tuturor liniile matricii se
executa:

Pentru toate serverele se verifica daca a venit vreun mesaj pe coada de read
si daca da se executa pentru serverul respectiv:
\begin{enumerate}
\item Citesc acest mesaj pentru a nu ramine in coada, el este ignorat.
\item Trimit liniile alocate serverului, una cite una cu ajutorul functiei
{\it write} care nu este o functie blocanta si are avantajul ca poate
trimite blocuri binare de date.
\item Trimit vectorul de linii (un vector in care se spune fiecare linie
care este in cadrul  matricii sistemului de ecuatii), se trimite tot cu
functia {\it write} deoarece stie sa transmita un vector intreg de date
float.
\item Scot din coada de read serverul respectiv si setez variabila din {\it
ready}.
\end{enumerate}

Din bucla while se iese in momentul in care toate serverele au primit liniile 
ce ii revin testind ca suma valorilor din vectorul {\it ready} sa fie egala
cu numarul de servere.

\pagebreak
Implementare eliminarii gauss:
\begin{enumerate}
\item Se receptioneaza numarul liniei.
\item Se receptioneaza numarul hostului care transmite linia.
\item Se receptioneaza linia intreaga.
\item Daca hostul urmator nu este hostul care a trimis linia sau daca
sistemul curent nu este cel care are ultima ecuatia si este la ultima
ecuatie (pentru calcul), transmite mai departe linia.
\item Se fac eliminarile pentru toate liniile ramase, nu numai pentru linia
curenta.
\item Daca mai sunt linii de receptionat pentru linia curenta (numarul linie
pentru care se face calculul -1) se reia de la pasul 1.
\item Se face calculul pentru linia curenta.
\item Cu rezultatul obtinut de la pasul alterior se fac eliminarile si
pentru liniile ramase deoarece nu se va mai primi acesta linie.
\item Se trimite numarul liniei.
\item Se trimite numarul hostului care transmite linia.
\item Se trimite linia intreaga.
\item Se reia de la pasul 1.
\end{enumerate}


\end{document}