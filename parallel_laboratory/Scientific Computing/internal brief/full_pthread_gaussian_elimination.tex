
\documentclass[a4paper]{article}
%%%%%%%%%%%%%%%%%%%%%%%%%%%%%%%%%%%%%%%%%%%%%%%%%%%%%%%%%%%%%%%%%%%%%%%%%%%%%%%%%%%%%%%%%%%%%%%%%%%%%%%%%%%%%%%%%%%%%%%%%%%%%%%%%%%%%%%%%%%%%%%%%%%%%%%%%%%%%%%%%%%%%%%%%%%%%%%%%%%%%%%%%%%%%%%%%%%%%%%%%%%%%%%%%%%%%%%%%%%%%%%%%%%%%%%%%%%%%%%%%%%%%%%%%%%%
%TCIDATA{OutputFilter=LATEX.DLL}
%TCIDATA{Version=5.00.0.2552}
%TCIDATA{<META NAME="SaveForMode" CONTENT="1">}
%TCIDATA{Created=Wednesday, September 07, 2005 13:30:08}
%TCIDATA{LastRevised=Saturday, September 10, 2005 18:37:13}
%TCIDATA{<META NAME="GraphicsSave" CONTENT="32">}
%TCIDATA{<META NAME="DocumentShell" CONTENT="Standard LaTeX\Blank - Standard LaTeX Article">}
%TCIDATA{Language=American English}
%TCIDATA{CSTFile=40 LaTeX article.cst}

\setlength{\textheight}{8in}
\setlength{\textwidth}{6in}
\setlength{\oddsidemargin}{0mm}
\setlength{\evensidemargin}{0mm}
\setlength{\marginparwidth}{0mm}
\setlength{\marginparsep}{0mm}
\newtheorem{theorem}{Theorem}
\newtheorem{acknowledgement}[theorem]{Acknowledgement}
\newtheorem{algorithm}[theorem]{Algorithm}
\newtheorem{axiom}[theorem]{Axiom}
\newtheorem{case}[theorem]{Case}
\newtheorem{claim}[theorem]{Claim}
\newtheorem{conclusion}[theorem]{Conclusion}
\newtheorem{condition}[theorem]{Condition}
\newtheorem{conjecture}[theorem]{Conjecture}
\newtheorem{corollary}[theorem]{Corollary}
\newtheorem{criterion}[theorem]{Criterion}
\newtheorem{definition}[theorem]{Definition}
\newtheorem{example}[theorem]{Example}
\newtheorem{exercise}[theorem]{Exercise}
\newtheorem{lemma}[theorem]{Lemma}
\newtheorem{notation}[theorem]{Notation}
\newtheorem{problem}[theorem]{Problem}
\newtheorem{proposition}[theorem]{Proposition}
\newtheorem{remark}[theorem]{Remark}
\newtheorem{solution}[theorem]{Solution}
\newtheorem{summary}[theorem]{Summary}
\newenvironment{proof}[1][Proof]{\noindent\textbf{#1.} }{\ \rule{0.5em}{0.5em}}
\input{tcilatex}

\begin{document}

\title{Pthread Implementation of Gaussian Elimination}
\date{}
\author{Gabriel Dimitriu}
\maketitle

\section{Introduction}

Let the system be 
\begin{equation}
Ax=b  \label{s1}
\end{equation}%
\qquad \qquad Here $A$ is a dense n x n matrix of coefficients such that $%
A[i][j]=a_{i,j}$, b is an n x 1 vector and x is the desired solution vector.

A system of equations is solved in two stages. First, through a series of
algebraic manipulations, the original system of equations is reduced to an
upper triangular system of the form $Ux=y$, where $U$ is a unit
upper-triangular matrix. In the second stage of solving a system of linear
equations, the upper-triangular system is solved for the variables in
reverse order from $x_{n-1}$ to $x_{0}$ by a procedure know as
back-substitution.

\begin{algorithm}
gaussian\_elimination
\end{algorithm}

\begin{enumerate}
\item procedure gaussian\_elimination(A,b,y)

\item begin

\item \qquad for k:=0 to n-1 do

\item \qquad \qquad for j:=k+1 to n-1 do

\item \qquad \qquad \qquad A[k][j]:=A[k][j]/A[k][k]; //division step

\item \qquad \qquad y[k]:=b[k]/A[k][k];

\item \qquad \qquad A[k][k]:=1;

\item \qquad \qquad for i:=k+1 to n-1 do

\item \qquad \qquad \qquad for j:=k+1 to n-1 do

\item \qquad \qquad \qquad \qquad A[i][j]:=A[i][j]-A[i]k]xA[k][j];
//elimination step

\item \qquad \qquad \qquad b[i]:=b[i]-A[i][k]xy[k];

\item \qquad \qquad \qquad A[i][k]:=0;

\item \qquad \qquad endfor;

\item \qquad endfor;

\item end gaussian\_elimination;
\end{enumerate}

\section{\protect\bigskip Pipeline Communication and Computation}

\bigskip The matrix is scattered to all processor so two consecutive row are
in two consecutive processors.

During the $k^{th}$ iteration processor $P_{k}$ broadcast part of the $k^{th}
$ row of the matrix to processors $P_{k+1},...,P_{p-1}$. Assume that the
processors $P_{0}...P_{p-1}$ are connected in a linear array, and $P_{k+1}$
is the first processor to receive the $k^{th}$ row from processor $P_{k}$.
Then the processor $P_{k+1}$ must forward this data to $P_{k+2}$. However,
after forwarding the $k^{th}$ row to $P_{k+2}$, processor $P_{k+1}$ need not
wait to perform the elimination step until all the processors up to $P_{p-1}$
have received the $k^{th}\,$\ row. Similarly, $P_{k+2}$ can start its
computation as soon as is has forwarded the $k^{th}$ row to $P_{k+3}$, and
so on. Meanwhile, after completing the computation for the $k^{th}$
iteration, $P_{k+1}$ can perform the division step, and start the broadcast
of the $(k+1)^{th}$ row by sending it to $P_{k+2}$. In this case of shared
memory the receive, send and broadcast from MPI\ are replaced by pthread
mutex procedures but without pipeline communication.

This algorithm is made for comparing the MPI, pthread and OpenMP
implementation.

\section{Results}

I have compiled the Gaussian Elimination without load balancing and pivoting
program with native gcc of the Fedora Core 1with maximum optimization "-O3".

The executable were run on a dual pentium II at 500MHz with 256MB RAM and
with LINUX\ Fedora Core 1.

The following result was made for a average or 10 runs for serial and
parallel programs.

\begin{center}
\FRAME{itbpF}{5.066in}{3.8057in}{0in}{}{}{Figure}{\special{language
"Scientific Word";type "GRAPHIC";maintain-aspect-ratio TRUE;display
"USEDEF";valid_file "T";width 5.066in;height 3.8057in;depth
0in;original-width 5.3756in;original-height 4.0315in;cropleft "0";croptop
"1";cropright "1";cropbottom "0";tempfilename
'gcc_pthread_ge_2500.bmp';tempfile-properties "XNPR";}}
\end{center}

\end{document}
