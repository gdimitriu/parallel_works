
\documentclass[a4paper]{article}
%%%%%%%%%%%%%%%%%%%%%%%%%%%%%%%%%%%%%%%%%%%%%%%%%%%%%%%%%%%%%%%%%%%%%%%%%%%%%%%%%%%%%%%%%%%%%%%%%%%%%%%%%%%%%%%%%%%%%%%%%%%%%%%%%%%%%%%%%%%%%%%%%%%%%%%%%%%%%%%%%%%%%%%%%%%%%%%%%%%%%%%%%%%%%%%%%%%%%%%%%%%%%%%%%%%%%%%%%%%%%%%%%%%%%%%%%%%%%%%%%%%%%%%%%%%%
%TCIDATA{OutputFilter=LATEX.DLL}
%TCIDATA{Version=5.00.0.2552}
%TCIDATA{<META NAME="SaveForMode" CONTENT="1">}
%TCIDATA{Created=Wednesday, September 07, 2005 13:30:08}
%TCIDATA{LastRevised=Sunday, September 11, 2005 14:04:18}
%TCIDATA{<META NAME="GraphicsSave" CONTENT="32">}
%TCIDATA{<META NAME="DocumentShell" CONTENT="Standard LaTeX\Blank - Standard LaTeX Article">}
%TCIDATA{Language=American English}
%TCIDATA{CSTFile=40 LaTeX article.cst}

\setlength{\textheight}{8in}
\setlength{\textwidth}{6in}
\setlength{\oddsidemargin}{0mm}
\setlength{\evensidemargin}{0mm}
\setlength{\marginparwidth}{0mm}
\setlength{\marginparsep}{0mm}
\newtheorem{theorem}{Theorem}
\newtheorem{acknowledgement}[theorem]{Acknowledgement}
\newtheorem{algorithm}[theorem]{Algorithm}
\newtheorem{axiom}[theorem]{Axiom}
\newtheorem{case}[theorem]{Case}
\newtheorem{claim}[theorem]{Claim}
\newtheorem{conclusion}[theorem]{Conclusion}
\newtheorem{condition}[theorem]{Condition}
\newtheorem{conjecture}[theorem]{Conjecture}
\newtheorem{corollary}[theorem]{Corollary}
\newtheorem{criterion}[theorem]{Criterion}
\newtheorem{definition}[theorem]{Definition}
\newtheorem{example}[theorem]{Example}
\newtheorem{exercise}[theorem]{Exercise}
\newtheorem{lemma}[theorem]{Lemma}
\newtheorem{notation}[theorem]{Notation}
\newtheorem{problem}[theorem]{Problem}
\newtheorem{proposition}[theorem]{Proposition}
\newtheorem{remark}[theorem]{Remark}
\newtheorem{solution}[theorem]{Solution}
\newtheorem{summary}[theorem]{Summary}
\newenvironment{proof}[1][Proof]{\noindent\textbf{#1.} }{\ \rule{0.5em}{0.5em}}
\input{tcilatex}

\begin{document}

\title{OpenMP Implementation of Gauss Jordan}
\date{}
\author{Gabriel Dimitriu}
\maketitle

\section{Introduction}

Let 
\begin{equation}
Ax=b  \label{s1}
\end{equation}%
be our system of linear equations, with $A=(a_{ij})_{i,j\in \{1,...,m\}}$, $%
b=(b_{1},...,b_{m})$ and $x=(x_{1},...,x_{m})^{T}$.

\begin{algorithm}
Gauss-Jordan
\end{algorithm}

\begin{enumerate}
\item procedure gauss\_jordan

\item for k=0 to m do

\item \qquad for i=k+1 to m do

\item \qquad \qquad for j=k+1 to m do

\item \qquad \qquad \qquad $a_{ij}=a_{ij}-a_{ik}\ast a_{kj}/a_{kk}$

\item \qquad \qquad endfor

\item \qquad \qquad $b_{i}=b_{i}-a_{ik}\ast b_{k}/a_{kk}$

\item \qquad endfor

\item \qquad for i=0 to k do

\item \qquad \qquad for j=k+1 to m do

\item \qquad \qquad \qquad $a_{ij}=a_{ij}-a_{ik}\ast a_{kj}/a_{kk}$

\item \qquad \qquad endfor

\item \qquad \qquad $b_{i}=b_{i}-a_{ik}\ast b_{k}/a_{kk}$

\item \qquad endfor

\item endfor

\item for i=0 to m do

\item \qquad $x_{i}=b_{i}/a_{ii}$

\item endfor

\item end procedure
\end{enumerate}

\section{Pipeline Communication and Computation}

\bigskip The matrix is scattered to all processor so two consecutive row are
in two consecutive processors.

During the $k^{th}$ iteration processor $P_{k}$ broadcast part of the $k^{th}
$ row of the matrix to processors $P_{k+1},...,P_{p-1}$. Assume that the
processors $P_{0}...P_{p-1}$ are connected in a linear array, and $P_{k+1}$
is the first processor to receive the $k^{th}$ row from processor $P_{k}$.
Then the processor $P_{k+1}$ must forward this data to $P_{k+2}$. However,
after forwarding the $k^{th}$ row to $P_{k+2}$, processor $P_{k+1}$ need not
wait to perform the elimination step until all the processors up to $P_{p-1}$
have received the $k^{th}\,$\ row. Similarly, $P_{k+2}$ can start its
computation as soon as is has forwarded the $k^{th}$ row to $P_{k+3}$, and
so on. Meanwhile, after completing the computation for the $k^{th}$
iteration, $P_{k+1}$ can perform the division step, and start the broadcast
of the $(k+1)^{th}$ row by sending it to $P_{k+2}$. In this case of shared
memory the receive, send and broadcast from MPI\ are replaced by OpenMP lock
procedures but without pipeline communication. The backward implementation
is made similar to the forward implementation.

\section{Results}

I have compile the parallel program with two OpenMP compilers: Omni 1.6 and
Intel C Compiler 8.0 for LINUX and the serial with gcc and Intel C Compiler
8.0 for LINUX both with maximum optimization "-O3" and for Intel C Compiler
I've put also "-mcpu=pentiumpro -tpp6" for maximum optimization.

The executable were run on a dual pentium II at 500MHz with 256MB RAM and
with LINUX\ Fedora Core 1.

The following results were made for a average or 10 runs for serial and
parallel programs and with red is plotted the results from ICC and with blue
the results from Omni.

\begin{center}
\FRAME{itbpF}{4.6666in}{3.506in}{0in}{}{}{Figure}{\special{language
"Scientific Word";type "GRAPHIC";maintain-aspect-ratio TRUE;display
"USEDEF";valid_file "T";width 4.6666in;height 3.506in;depth
0in;original-width 5.3756in;original-height 4.0315in;cropleft "0";croptop
"1";cropright "1";cropbottom "0";tempfilename
'full_openmp_gauss_jordan_2500.bmp';tempfile-properties "XNPR";}}
\end{center}

\end{document}
