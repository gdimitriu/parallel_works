\chapter{Instructiuni de utilizare}
\section{Configurarea retelei}

\hspace{5mm}Serverele pe sistemele host se pornesc cu comanda:

$$gaussd\; <port>$$

Daca este specificat portul se porneste standard pe 50000.

De remarcat ca se pot porni mai multe servere pe accesi masina, bineinteles
cu porturi diferite.

Se editeaza fisierul {\it gauss.conf} care contine adresele de IP si
porturile serverelor care sunt valide. Serverele se definesc ca IP:port.

De exemplu:

\input{gauss.conf}

\section{Introducerea sistemului de ecuatii}

\hspace{5mm} Dupa lansare in executie a programului gauss si dupa ce inelul a
fost stabilit se poate introduce numele fisierului care contine sistemul de
ecuatii.

Sistemul de ecuatii este descis in forma citibila de om, adica forma
matematica de descriere a unui sistem de ecuatii($aX1+bX2=c$).

{\bf Observatie:} Daca un coeficient este zero se introduce acest coeficient
zero nu se exclude variabila.

Un exemplu de sistem de ecuatii pe care s-au facut testele  de corectitudine
a calculelor este:

\pagebreak

\input{test3.dat}

Se observa ca este un sistem cu 11 necunoscute care are solutiile:
$$\begin{array}{c}
X_1=1\\
X_2=2\\
X_3=3\\
X_4=1\\
X_5=2\\
X_6=2\\
X_7=1\\
X_8=1\\
X_9=1\\
X_{10}=2\\
X_{11}=1\\
\end{array}$$

\section{Concluzii}

\hspace{5mm}Sistemul a fost testat pentru un numar relativ mic de ecuatii
(30 de necunoscute) rulind pe trei sisteme (2 + clientul-1server) in 1.70
secunde, cele trei sisteme au fost: doua pentiumuri la 233 si un pentium la
100. Algoritmul secvential a fost testat pe un pentium la 120 si a rulat
sub 1 secunda. Deci putem spune ca nu avem accelerare si o decelerare pentru 
ecuatii cu un numar mic de necunoscute din cauza transmiterii liniilor pe o 
retea LAN lenta. 

Daca se doreste accelerare ar trebui sa avem un sistem care are o retea 
interna de mare viteza, astfel incit timpul de transmisie prin retea sa fie 
destul de mic in camparatie cu calculele care se fac. Sau sa avem un sistem 
de ecuatii foarte mare (pe o singura masina s-a verificat un sistem de ecuatii
 cu 100 de necunoscute rulind (pentiun 120) in aproximativ 11 de secunde, pe 
cind pe acelasi calculator, acealasi sistem de ecuatii a fost rulat in 3.75 secunde 
cu algoritmul secvential deci se poate observa o incetinire de 3 ori. Emularea
 a fost facuta cu trei servere.

Algoritmul este posibil sa aiba accelerare si pe sisteme distribuite daca
sistemul est foarte mare, astfel incit un singur sistem sa trebuiasca sa
iasa pe swap, iar in cazul implentarii distribuite nici un sistem sa nu iasa
pe swap. Dar aici se pune acceasi problema a transmiterii de linii prin retea
care bineinteles creste si ea. In acest fel este posibil sa nu avem
accelerare de asemenea. Pe sisteme distribuite este posibil sa nu se poata
realiza acest algoritm deoarece este nesara schimbul de linii prin retea cu
confirmari. Daca se gaseste o solutie in care sa nu fie necesara
"confirmarea" din partea receptorului si sa avem o retea rapida este posibil
sa avem accelerare.

De remarcat ca daca se face schimbarea pivotului in intreaga retea,
realizind o cautare paralela si pipeline am putea avea din accelerare
pozitiva (in cel mai bun caz) sau unitara.