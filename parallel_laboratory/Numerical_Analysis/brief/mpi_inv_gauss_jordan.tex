
\documentclass[a4paper]{article}
%%%%%%%%%%%%%%%%%%%%%%%%%%%%%%%%%%%%%%%%%%%%%%%%%%%%%%%%%%%%%%%%%%%%%%%%%%%%%%%%%%%%%%%%%%%%%%%%%%%%%%%%%%%%%%%%%%%%%%%%%%%%%%%%%%%%%%%%%%%%%%%%%%%%%%%%%%%%%%%%%%%%%%%%%%%%%%%%%%%%%%%%%%%%%%%%%%%%%%%%%%%%%%%%%%%%%%%%%%%%%%%%%%%%%%%%%%%%%%%%%%%%%%%%%%%%
%TCIDATA{OutputFilter=LATEX.DLL}
%TCIDATA{Version=5.00.0.2552}
%TCIDATA{<META NAME="SaveForMode" CONTENT="1">}
%TCIDATA{Created=Wednesday, August 24, 2005 17:13:14}
%TCIDATA{LastRevised=Sunday, November 06, 2005 11:40:48}
%TCIDATA{<META NAME="GraphicsSave" CONTENT="32">}
%TCIDATA{<META NAME="DocumentShell" CONTENT="Standard LaTeX\Blank - Standard LaTeX Article">}
%TCIDATA{Language=American English}
%TCIDATA{CSTFile=40 LaTeX article.cst}

\setlength{\textheight}{8in}
\setlength{\textwidth}{6in}
\setlength{\oddsidemargin}{0mm}
\setlength{\evensidemargin}{0mm}
\setlength{\marginparwidth}{0mm}
\setlength{\marginparsep}{0mm}
\newtheorem{theorem}{Theorem}
\newtheorem{acknowledgement}[theorem]{Acknowledgement}
\newtheorem{algorithm}[theorem]{Algorithm}
\newtheorem{axiom}[theorem]{Axiom}
\newtheorem{case}[theorem]{Case}
\newtheorem{claim}[theorem]{Claim}
\newtheorem{conclusion}[theorem]{Conclusion}
\newtheorem{condition}[theorem]{Condition}
\newtheorem{conjecture}[theorem]{Conjecture}
\newtheorem{corollary}[theorem]{Corollary}
\newtheorem{criterion}[theorem]{Criterion}
\newtheorem{definition}[theorem]{Definition}
\newtheorem{example}[theorem]{Example}
\newtheorem{exercise}[theorem]{Exercise}
\newtheorem{lemma}[theorem]{Lemma}
\newtheorem{notation}[theorem]{Notation}
\newtheorem{problem}[theorem]{Problem}
\newtheorem{proposition}[theorem]{Proposition}
\newtheorem{remark}[theorem]{Remark}
\newtheorem{solution}[theorem]{Solution}
\newtheorem{summary}[theorem]{Summary}
\newenvironment{proof}[1][Proof]{\noindent\textbf{#1.} }{\ \rule{0.5em}{0.5em}}
\input{tcilatex}

\begin{document}

\title{MPI implementation of inverse with Gauss Jordan}
\date{}
\author{Gabriel Dimitriu}
\maketitle

\section{\protect\bigskip Introduction}

\bigskip Let be $A$ an inversable matrix and his inverse is $A^{-1}$, then
the inverse have the following proprety $A\ast A^{-1}=I$, where $I$ is the
unity matrix.

Through a series of algebraic manipulations, matrix is reduced to a unity
matrix and making the same operations for an attached unity matrix. The
former unity matrix will be the inverse.

\begin{algorithm}
Inverse with Gauss Jordan
\end{algorithm}

\begin{enumerate}
\item procedure inv\_gj(mat,inv)

\item begin

\item \qquad for k:=0 to n-1 do

\item \qquad \qquad for i:=0 to n-1 do

\item \qquad \qquad \qquad if(k\TEXTsymbol{<}\TEXTsymbol{>}i) then do

\item \qquad \qquad \qquad \qquad for j:=k+1 to n-1 do

\item \qquad \qquad \qquad \qquad \qquad
mat[i][j]-=mat[i][k]/mat[k][k]*mat[k][j];

\item \qquad \qquad \qquad \qquad for j:=0 to n-1 do

\item \qquad \qquad \qquad \qquad \qquad
inv[i][j]-=mat[i][k]/mat[k][k]*inv[k][j];

\item \qquad \qquad \qquad else

\item \qquad \qquad \qquad \qquad for j:=0 to n-1 do

\item \qquad \qquad \qquad \qquad \qquad inv[i][j]=inv[i][j]/mat[i][i];

\item \qquad \qquad \qquad \qquad \qquad if j\TEXTsymbol{<}\TEXTsymbol{>}i
then do

\item \qquad \qquad \qquad \qquad \qquad \qquad
mat[i][j]=mat[i][j]/mat[i][i];

\item \qquad \qquad \qquad \qquad endfor;

\item \qquad \qquad \qquad \qquad mat[i][i]=1.0;

\item \qquad \qquad \qquad endif;

\item \qquad \qquad endfor;\qquad 

\item \qquad endfor;

\item end inv\_gj;
\end{enumerate}

\section{Pipeline communication and Computation}

\bigskip The matrix is scattered to all processor so two consecutive row are
in two consecutive processors.

During the $k^{th}$ iteration processor $P_{k}$ broadcast the $k^{th}$ rows
of both matrix to processors all processors. Assume that the processors $%
P_{0}...P_{p-1}$ are connected in a linear array, and $P_{k+1}$ is the first
processor to receive the $k^{th}$ rows from processor $P_{k}$. Then the
processor $P_{k+1}$ must forward this data to $P_{k+2}$. However, after
forwarding the $k^{th}$ rows to $P_{k+2}$, processor $P_{k+1}$ need not wait
to perform the for with i variable for it's part of matrix until all the
processors up to $P_{p-1}$ have received the $k^{th}\,$\ rows. Similarly, $%
P_{k+2}$ can start its computation as soon as is has forwarded the $k^{th}$
rows to $P_{k+3}$, and so on. Meanwhile, after completing the computation
for the $k^{th}$ iteration, $P_{k+1}$ start the broadcast of the $(k+1)^{th}$
rows by sending it to $P_{k+2}$. All processors receive p-1 rows before
sending their rows to other processors.

\section{\protect\bigskip Results}

The benchmark is made on a 100MBs network with Fedora Core 1 and realteck
8139 network card. The MPI version is MPICH\ 1.2.6 with native gcc and rsh
authorization.

The network consists in 1 PII-500MHz, 2 PIII-800MHz, 1 athlon-1200MHz, 1
PIV-1400MHz and 1 dual PII-500MHz and the speedup is the average of ten runs.

The Inverse with Gauss Jordan is implemented without load balancing so
because the fist and in the same time the slowest computer is PII-500MHz so
the reference of the speedup is this computer.

In the following graph is presented the speedup with red for two computers,
blue for three computers, green for four computers, cyan for five computers,
magenta for six computers and black for seven computers.

\begin{center}
\FRAME{itbpF}{4.3412in}{3.2619in}{0in}{}{}{Figure}{\special{language
"Scientific Word";type "GRAPHIC";maintain-aspect-ratio TRUE;display
"USEDEF";valid_file "T";width 4.3412in;height 3.2619in;depth
0in;original-width 5.3756in;original-height 4.0315in;cropleft "0";croptop
"1";cropright "1";cropbottom "0";tempfilename
'mpi_inv_gj_1500.bmp';tempfile-properties "XNPR";}}
\end{center}

\end{document}
